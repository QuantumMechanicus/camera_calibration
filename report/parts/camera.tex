\label{model}
В статье \cite{Fitzgibbon01b} была предложена следующая $n$-параметрическая модель камеры с радиальной дисторсией. Пусть $u$ --- точка на распрямленном изображении, $p$ --- соответствующая точка с дисторсией, связанные следующим соотношением:
\begin{equation}
	u \sim \frac{1}{1 + \lambda_1\norm{p}^2 + \lambda_2\norm{p}^4 + \cdots + \lambda_n\norm{p}^{2n}} = U(p) \label{eq:division_model}
\end{equation} 
Эта модель именуется моделью с делением. Это основная модель для нашей задачи (см. \href{https://github.com/QuantumMechanicus/camera_calibration/blob/dev/core/scene/Intrinsics.h}{реализацию}). В основном мы будем пользоваться ей при $n = 1, 2$.

\section{Общая постановка задачи}
Пусть имеется точка 3D мира $w$, тогда проекция в пространство камеры происходит следующим образом:
\begin{itemize}
	\item Осуществим переход в локальные координаты камеры с помощью преобразования \\$[R_w$ $t_w]$ --- вращения и переноса относительно мировой системы координат.
	\item Рассмотрим матрицу калибровки следующего вида: 
	\begin{equation}
		C = \begin{bmatrix}
		f & 0 & p_x \\
		0 & f & p_y \\
		0 & 0 & 1 \\
		\end{bmatrix}, 
	\end{equation}
	где $f$ --- фокусное расстояние,  $\left(p_x, p_y\right)$ --- координаты принципиальной точки. Эта матрица осуществляет преобразование из 3D мира в однородные 2D координаты изображения без дисторсии: $Cw \sim u$. 
	\item На этом этапе работает эпиполярная геометрия, поэтому важным для калибровки понятием является фундаментальная матрица, связывающая пару изображений $F = C^{-T}EC^{-1}$, где $E = R [t]_{\times}$ --- существенная матрица, $[R$ $t]$ ---  относительное движение между камерами, $[t]_{\times}$ --- матричное представление операции векторного произведения с $t$.   
	\item Произведем преобразование дисторсии: $D(u) \sim p$, где $D$ --- обратное к \eqref{eq:division_model}, а $p$ будет точкой на входном изображении. TODO
\end{itemize} 
Таким образом, для решения задачи калибровки необходимо в первую очередь определить преобразование $D$ (для этого достаточно найти коэффициенты $\lambda_1$, \dots, $\lambda_n$ и искать корни соответствующего уравнения). Затем, сделав преоразование $U$, можно получить распрямленные изображения, с которыми работают в терминах фундаментальных и калибровочных матриц. Решением задачи в таком случае будут оценки для коэффициентов дисторсии, фокусного расстояния, координат принципиальной точки и преобразований между локальными координатами камер.
