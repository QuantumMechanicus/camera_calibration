\label{nonlinear}
\section{Постановка задачи}
Зафиксируем некоторое количество коэффициентов дисторсии. Для определенности будем рассматривать модель с двумя коэффициентами (в реализации это произвольно задаваемый параметр) --- $\lambda_1, 
\lambda_2$. Параметризуем фундаментальную матрицу как набор из 8 чисел (правое нижнее число фиксируем как единицу), удовлетворяющих свойству неполного ранга:
\begin{equation}
	F = \begin{bmatrix}
	f_{1,1} & f_{1,2} & f_{1,3} \\
	f_{2,1} & f_{2,2} & f_{2,3} \\
	f_{3,1} & f_{3,2} & 1 \\
	\end{bmatrix}, \rank F = 2.
\end{equation}
Чтобы сохранять инвариант ранга матрицы в нелинейной оптимизации, будем производить SVD-разложение $F$, обнулять наименьшее сингулярное число, а затем собирать матрицу обратно.

Обозначим $I$ --- ключевые точки-инлаеры изображения с дисторсией для заданного начального приближения. Рассмотрим следующую задачу минимизации: 
\begin{equation}
	\min\limits_{F, \lambda_1, \lambda_2} \frac{1}{2}\sum\limits_{\left(p_l, p_r\right) \in I} d\left(p_l, c_l\right)^2 + d\left(p_r, c_r\right)^2
\end{equation}

\section{Реализация}
В качестве начального приближения берутся результаты, полученные автоматическим решателем (неоцененные коэффициенты дисторсии считаем нулями), а затем \href{https://github.com/QuantumMechanicus/camera_calibration_test/blob/dev/subroutines/non_linear_optimizer/Non_Linear_Estimator.cpp}{воспользуемся} библиотекой \cite{ceres-solver}. 