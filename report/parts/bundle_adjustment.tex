\label{bundle_adj}
\section{Постановка задачи}
\href{https://github.com/QuantumMechanicus/camera_calibration_test/blob/dev/subroutines/global_non_linear_optimizer/Global_Non_Linear_Estimator.h#L30}{Функтор оптимизации} в данной задаче представляет из себя ошибку репроекции, мы оптимизируем по следующему набору параметров:
\begin{itemize}
	\item Матрица вращения $R$.
	\item Направление перемещения $t$.
	\item Коэффициенты дисторсии $\lambda_1,...,\lambda_n$.
	\item Координаты принципиальной точки $p_x, p_y$.
	\item Триангулированные пары инлаеров (в смысле ошибки до эпиполярной кривой), отфильтрованные по ограничению хиральности.
\end{itemize}	 
\subsection{Параметризация сферы}
Cфера \href{https://github.com/QuantumMechanicus/camera_calibration_test/blob/dev/core/utils/Local_Parametrization_Sphere.h}{параметризуется} следующим образом:
\begin{itemize}
	\item Пусть есть точка на сфере $t$ и соответствующий вектор из центра сферы $\bar{t}$.
	\item Найдем базис касательного пространства в этой точке. Для этого необходимо построить из $\bar{t}$ ортонормальную тройку векторов. Это можно сделать с помощью векторного произведения $\bar{t}$ с одной из осей $XYZ$. Выберем самую неколлинеарную относительно $\bar{t}$ ось. 
	\item Чтобы осуществить маленькое приращение перемещения, возьмем полученный базис и линейно сдвинемся в нем.
	\item Нормализуем полученную $\Delta t$, чтобы остаться на сфере.    
\end{itemize} 
\pagebreak
\subsection{Параметризация SO(3)}
Элемент $R \in SO(3)$ \href{https://github.com/QuantumMechanicus/camera_calibration_test/blob/dev/core/utils/Local_Parametrization_SO3.h}{параметризуется}  следующим образом:
\begin{itemize}
	\item Параметризуем $R$ как единичный кватернион $w$.
	\item Пусть $\omega$ это трехмерный вектор такой, что его направление это ось вращения $R$, а $\norm{\omega} = \theta$ --- угол поворота.
	\item Рассмотрим $[\omega]_{\times}$ --- кососимметричную матрицу векторного умножения на $\omega$.
	\item Тогда $\exp{([\omega]_{\times})} = R$.
	\item Чтобы осуществить маленькое приращение вращения, найдем по кватерниону $\omega$. Для $\omega$ как элемента линейного пространства осуществим приращение $\Delta\omega$, а затем найдем соответствующую матрицу вращения $\Delta R$, по которой получим новый кватернион $\Delta w$.    
\end{itemize}

\section{Реализация}
Реализацию можно увидеть \href{https://github.com/QuantumMechanicus/camera_calibration_test/tree/dev/subroutines/global_non_linear_optimizer}{здесь}.