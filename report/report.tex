\documentclass[oneside,openany,
a4paper,11pt]{report}
\usepackage{cite}
\usepackage{hyperref}
\usepackage{graphicx}
\usepackage{subcaption}
\usepackage{amsmath}
\usepackage{amssymb}
\usepackage{tikz}
\usepackage{verbatim}
\usepackage{bbold}
\usepackage{lastpage}
\usepackage{fancyhdr}
\usepackage{etoc}
\usepackage{listings}
\usepackage{indentfirst}
\usepackage{float}
\usepackage{multirow}
\usepackage{ctable}
\usepackage[utf8]{inputenc}
\usepackage[russian]{babel}
\usepackage[inner=2.5cm,outer=2cm,top=3cm,bottom=2.5cm]{geometry}
\pagestyle{fancy}
\graphicspath{{figures/}}
\renewcommand{\etocaftertitlehook}{\pagestyle{fancy}}
\renewcommand{\etocaftertochook}{\thispagestyle{fancy}}
\usepackage{titling}

\fancypagestyle{plain}{%
	\fancyfoot[R]{\today}
	\fancyfoot[L]{\thetitle \\ E-Mail: \href{mailto:Daniil.Bordinyuk@lanit-tercom.com}{Daniil.Bordinyuk@lanit-tercom.com}}
	\fancyfoot[C]{Page \thepage\ of \pageref{LastPage}}
	\fancyhead[L]{\textbf{\textit{PMVLab.} Ltd}}
	\fancyhead[R]{\leftmark}
}
\fancyfoot[R]{\today}
\fancyfoot[L]{\thetitle \\ E-Mail: \href{mailto:Daniil.Bordinyuk@lanit-tercom.com}{Daniil.Bordinyuk@lanit-tercom.com}}
\fancyfoot[C]{Page \thepage\ of \pageref{LastPage}}
\fancyhead[L]{\textbf{\textit{PMVLab.} Ltd}}
\fancyhead[R]{\leftmark}


\renewcommand{\headrulewidth}{0pt}
\renewcommand{\footrulewidth}{1pt}
%\usepackage{subfig}
\usetikzlibrary{arrows,shapes,positioning,shapes.multipart}
\usepackage{color}

\definecolor{mygreen}{rgb}{0,0.6,0}
\definecolor{mygray}{rgb}{0.5,0.5,0.5}
\definecolor{mymauve}{rgb}{0.58,0,0.82}

\lstset{ %
	backgroundcolor=\color{white},   
	basicstyle=\footnotesize,        
	breakatwhitespace=false,         
	breaklines=true,                 
	captionpos=b,                    
	commentstyle=\color{mygreen},    
	deletekeywords={...},            
	escapeinside={\%*}{*)},          
	extendedchars=true,              
	frame=single,	                 
	keepspaces=true,                 
	keywordstyle=\color{blue},       
	language=Octave,                 
	morekeywords={*,...},            
	numbers=left,                    
	numbersep=5pt,                   
	numberstyle=\tiny\color{mygray}, 
	rulecolor=\color{black},         
	showspaces=false,                
	showstringspaces=false,          
	showtabs=false,                  
	stepnumber=2,                    
	stringstyle=\color{mymauve},     
	tabsize=2,	                   % 
	title=\lstname                   
}
%\setcounter{tocdepth}{1}

\DeclareMathOperator*{\argmin}{\arg\!\min}
\DeclareMathOperator{\rank}{rank}
\newcommand{\norm}[1]{\left\lVert#1\right\rVert}

\renewenvironment{abstract}
 {\small
\begin{center}
	\bfseries \abstractname\vspace{-.5em}\vspace{0pt}
\end{center}
\list{}{%
	\setlength{\leftmargin}{15mm}% <---------- CHANGE HERE
			\setlength{\rightmargin}{\leftmargin}%
			}%
		\item\relax}
	{\endlist}
\newfloat{lstfloat}{htbp}{lop}
\floatname{lstfloat}{Listing}






\title{Здесь должно быть название}
\author{Даниил Бордынюк}

\begin{document}
\maketitle

\tableofcontents
\clearpage

\chapter*{Введение}
\markboth{\uppercase{Введение}}{}
\addcontentsline{toc}{chapter}{Введение}
Здесь должно быть небольшое введение.

\section{Используемые инструменты}
\begin{itemize}
	\item Основной язык разработки --- C++.
	\item Библиотека работы с матрицами --- Eigen \cite{eigen}. 
	\item Реализация групп Ли --- Sophus \cite{sophus}.
	\item Работа с изображениями --- openCV \cite{opencv_library}.
	\item Решение задачи наименьших квадратов --- ceres-solver \cite{ceres-solver}.
	\item Распараллеливание задач --- Intel® TBB \cite{tbb}.
	\item Тестирование и логгирование (на стадии добавления) --- glog \cite{glog} и gtest \cite{gtest}.
\end{itemize}

\section{Описание работы}
Здесь должно быть краткое описание работы.
	
\chapter{Модель камеры}
\label{model}
В статье \cite{Fitzgibbon01b} была предложена следующая $n$-параметрическая модель камеры с радиальной дисторсией. Пусть $u$ --- точка на распрямленном изображении, $p$ --- соответствующая точка с дисторсией, связанные следующим соотношением:
\begin{equation}
	u \sim \frac{1}{1 + \lambda_1\norm{p}^2 + \lambda_2\norm{p}^4 + \cdots + \lambda_n\norm{p}^{2n}} = U(p) \label{eq:division_model}
\end{equation} 
Эта модель именуется моделью с делением. Это основная модель для нашей задачи (см. \href{https://github.com/QuantumMechanicus/camera_calibration/blob/dev/core/scene/Intrinsics.h}{реализацию}). В основном мы будем пользоваться ей при $n = 1, 2$.

\section{Общая постановка задачи}
Пусть имеется точка 3D мира $w$, тогда проекция в пространство камеры происходит следующим образом:
\begin{itemize}
	\item Осуществим переход в локальные координаты камеры с помощью преобразования \\$[R_w$ $t_w]$ --- вращения и переноса относительно мировой системы координат.
	\item Рассмотрим матрицу калибровки следующего вида: 
	\begin{equation}
		C = \begin{bmatrix}
		f & 0 & p_x \\
		0 & f & p_y \\
		0 & 0 & 1 \\
		\end{bmatrix}, 
	\end{equation}
	где $f$ --- фокусное расстояние,  $\left(p_x, p_y\right)$ --- координаты принципиальной точки. Эта матрица осуществляет преобразование из 3D мира в однородные 2D координаты изображения без дисторсии: $Cw \sim u$. 
	\item На этом этапе работает эпиполярная геометрия, поэтому важным для калибровки понятием является фундаментальная матрица, связывающая пару изображений $F = C^{-T}EC^{-1}$, где $E = R [t]_{\times}$ --- существенная матрица, $[R$ $t]$ ---  относительное движение между камерами, $[t]_{\times}$ --- матричное представление операции векторного произведения с $t$.   
	\item Произведем преобразование дисторсии: $D(u) \sim p$, где $D$ --- обратное к \eqref{eq:division_model}, а $p$ будет точкой на входном изображении. TODO
\end{itemize} 
Таким образом, для решения задачи калибровки необходимо в первую очередь определить преобразование $D$ (для этого достаточно найти коэффициенты $\lambda_1$, \dots, $\lambda_n$ и искать корни соответствующего уравнения). Затем, сделав преоразование $U$, можно получить распрямленные изображения, с которыми работают в терминах фундаментальных и калибровочных матриц. Решением задачи в таком случае будут оценки для коэффициентов дисторсии, фокусного расстояния, координат принципиальной точки и преобразований между локальными координатами камер.


\chapter{Автоматический решатель}
\label{groebner}
\section{Постановка задачи}

Выше сформулированная модель с одним коэффициентом хорошо подходит для использования генератора\footnote{ Имплементацию генерирующую код на Matlab можно найти \href{https://github.com/PavelTrutman/Automatic-Generator}{здесь}.} решателей систем уравнений на основе базисов Грёбнера. Подробная постановка задачи рассмотрена в \cite[p.~121-149]{KukelovaPhD}


\section{Реализация}

Полученный \href{https://github.com/QuantumMechanicus/camera_calibration/blob/dev/subroutines/distortion_groebner_estimator/Groebner_Estimator.cpp#L125}{решатель} оборачивается в RANSAC, в котором мы минимизируем заранее заданный в качестве параметра квантиль выборки ошибок (в смысле \href{https://github.com/QuantumMechanicus/camera_calibration/blob/dev/core/utils/Functors.h#L11}{расстояния} до эпиполярной кривой), то есть функтор ошибки выглядит следующим образом:
\begin{enumerate}
	\item Делаем обратное преобразование дисторсии с помощью оцененного коэффициента для всех ключевых точек изображения с дисторсией ($p_l$ --- точка на левом изображении, $p_r$ --- на правом).
	\item С помощью оцененной фундаментальной  матрицы считаем левые и правые эпилинии для распрямленных точек.
	\item Находим ближайшие точки на соотвествуюших прямых.
	\item Применяем для них преобразование дисторсии, это дает точки на эпиполярных кривых $c_l$ и $c_r$.
	\item Считаем $d\left(p_l, c_l\right)$ и $d\left(p_r, c_r\right)$--- нормы разности для исходной точки и точки на кривой (это является оценкой расстояния до эпиполярной кривой от исходной точки при оцененных параметрах). 
\end{enumerate}

Найденные таким образом оценки фундаментальной матрицы и фокусного расстояния могут быть использованы для дальнейшей оптимизации и калибровки. 

\chapter{Нелинейная оптимизация}
\label{nonlinear}
\section{Постановка задачи}
Зафиксируем некоторое количество коэффициентов дисторсии. Для определенности будем рассматривать модель с двумя коэффициентами (в реализации это произвольно задаваемый параметр) --- $\lambda_1, 
\lambda_2$. Параметризуем фундаментальную матрицу как набор из 8 чисел (правое нижнее число фиксируем как единицу), удовлетворяющих свойству неполного ранга:
\begin{equation}
	F = \begin{bmatrix}
	f_{1,1} & f_{1,2} & f_{1,3} \\
	f_{2,1} & f_{2,2} & f_{2,3} \\
	f_{3,1} & f_{3,2} & 1 \\
	\end{bmatrix}, \rank F = 2.
\end{equation}
Чтобы сохранять инвариант ранга матрицы в нелинейной оптимизации, будем производить SVD-разложение $F$, обнулять наименьшее сингулярное число, а затем собирать матрицу обратно.

Обозначим $I$ --- ключевые точки-инлаеры изображения с дисторсией для заданного начального приближения. Рассмотрим следующую задачу минимизации: 
\begin{equation}
	\min\limits_{F, \lambda_1, \lambda_2} \frac{1}{2}\sum\limits_{\left(p_l, p_r\right) \in I} d\left(p_l, c_l\right)^2 + d\left(p_r, c_r\right)^2
\end{equation}

\section{Реализация}
В качестве начального приближения берутся результаты, полученные автоматическим решателем (неоцененные коэффициенты дисторсии считаем нулями), а затем \href{https://github.com/QuantumMechanicus/camera_calibration_test/blob/dev/subroutines/non_linear_optimizer/Non_Linear_Estimator.cpp}{воспользуемся} библиотекой \cite{ceres-solver}. 

\chapter{Оценка фокусных расстояний}
\label{focal}
\section{Постановка задачи}
Из оцененных фундаментальных матриц можно извлечь фокусное расстояние. Один из методов это сделать описан в статье \cite{Sturm:2005:FLC:1090456.1649082}. В нем строятся два линейных и одно квадратное уравнение относительно квадрата фокусного расстояния --- $f^2$, обозначим эти уравнения за $a\left(f^2\right), b\left(f^2\right), h\left(f^2\right)$. Этот метод позволяет получить  оценку фокусного расстояния для каждой пары в отдельности, мы несколько модифицируем его. 

Пусть у нас есть $F_1, \cdots F_k$ --- оценки фундаментальных матриц для $k$ пар, и из них получены $a_i, b_i, h_i$. 
Рассмотрим:
\begin{equation}
	h^2 = \sum\limits_{i=1}^{k} h^2_i.
\end{equation} 
и уравнение 
\begin{equation}
	\frac{\mathrm{d} h^2}{\mathrm{d}f^2} = 2hh' = 0.
\end{equation} 
Очевидно, при точных оценках на $F_i$  одним из его корней является $f^2$. Найдем все положительные (строго отделенные от нуля) вещественные корни такого уравнения и выберем из них тот, который ко всему прочему минимизирует сумму $\sum\limits_{i=1}^{k} a_i^2 + b_i^2$. Этот подход позволяет оценить $f^2$ сразу совместно по всем фундаментальным матрицам, однако требует, чтобы все оценки $F_k$ были достаточно близкими к правде. Если этого не получается обеспечить, то надо использовать какую-нибудь робастную оценку типа медианы для выборки из оценок $f^2$ для каждой пары.
\section{Реализация}
Реализацию можно увидеть \href{https://github.com/QuantumMechanicus/camera_calibration_test/blob/dev/subroutines/focal_length_estimator/Focal_Estimator.cpp}{здесь}.


\chapter{Bundle adjustment}
\section{Постановка задачи}

\subsection{Параметризация сферы}

\subsection{Параметризация SO(3)}

\subsection{Функтор ошибки}

\section{Реализация}

\chapter{Некоторые результаты}
\section{Оценки коэффициентов дисторсии}
Таблица с оценками
\subsection{Примеры распрямленных изображений}
3 картинки для 3 различных камер
\section{Оценки фундаментальных матриц}
3 картинки с инлайерами и эпилиниями
\section{Оценки поз}
Траектория движения
\subsection{Примеры облаков триангулированных точек}
Облака точек


\bibliographystyle{ugost2008ls}
\bibliography{parts/refs.bib}
\end{document}
